% LaTeX Curriculum Vitae Template
%
% Copyright (C) 2004-2009 Jason Blevins <jrblevin@sdf.lonestar.org>
% http://jblevins.org/projects/cv-template/
%
% You may use use this document as a template to create your own CV
% and you may redistribute the source code freely. No attribution is
% required in any resulting documents. I do ask that you please leave
% this notice and the above URL in the source code if you choose to
% redistribute this file.

\documentclass[a4paper, 10pt]{article}

\usepackage{hyperref}
\usepackage[hmargin=2.5cm,vmargin=2.5cm]{geometry}
\usepackage{graphicx}

% Comment the following lines to use the default Computer Modern font
% instead of the Palatino font provided by the mathpazo package.
% Remove the 'osf' bit if you don't like the old style figures.
%\usepackage[T1]{fontenc}
%\usepackage[sc]{mathpazo}

% Set your name here
\def\name{Casimir Ludwig}

% Replace this with a link to your CV if you like, or set it empty
% (as in \def\footerlink{}) to remove the link in the footer:
\def\footerlink{}

% The following metadata will show up in the PDF properties
\hypersetup{
  colorlinks = true,
  urlcolor = black,
  pdfauthor = {\name},
  pdfkeywords = {psychology, vision, eye movement control, computational modelling},
  pdftitle = {\name: Curriculum Vitae},
  pdfsubject = {Curriculum Vitae},
  pdfpagemode = UseNone
}

%\geometry{
%  body={6.5in, 8.5in},
%  left=1.0in,
%  top=1.25in
%}

% Customize page headers
\pagestyle{myheadings}
\markright{\name}
\thispagestyle{empty}

% Custom section fonts
\usepackage{sectsty}
\sectionfont{\rmfamily\mdseries\Large}
\subsectionfont{\rmfamily\mdseries\itshape\large}

% Other possible font commands include:
% \ttfamily for teletype,
% \sffamily for sans serif,
% \bfseries for bold,
% \scshape for small caps,
% \normalsize, \large, \Large, \LARGE sizes.

% Don't indent paragraphs.
\setlength\parindent{0em}

% Make lists without bullets
\renewenvironment{itemize}{
  \begin{list}{}{
    \setlength{\leftmargin}{1.5em}
  }
}{
  \end{list}
}

\begin{document}

% Place name at left
{\huge \name}

% Alternatively, print name centered and bold:
%\centerline{\huge \bf \name}

\vspace{0.1in}

\begin{minipage}{0.45\linewidth}
  \href{http://www.bris.ac.uk/}{University of Bristol} \\
  School of Psychological Science \\
  12a Priory Road \\
  Bristol BS8 1TU\\
  United Kingdom
\end{minipage}
\begin{minipage}{0.45\linewidth}
  \begin{tabular}{ll}
    Phone: & (0044) (0) 117 33 x17251\\
    Fax: &  (0044) (0) 117 928 8588\\
    Email: & \href{mailto:c.ludwig@bristol.ac.uk}{c.ludwig@bristol.ac.uk} \\
    Homepage: & \href{https://casludwig.github.io/}{https://casludwig.github.io/} \\
  \end{tabular}
\end{minipage}


%\section*{Personal}
%
%\begin{itemize}
%  \item Date of birth: 27-09-1974.
%  \item Nationality: Dutch.
%\end{itemize}

\section*{Academic qualifications}
\begin{itemize}
  \item PhD Experimental Psychology, University of Bristol, 2003.
    \begin{itemize}
    \item \textit{Thesis title:}
      Stimulus-driven and goal-driven control over visual selection.
%    \item \textit{Supervisor:}
%      Professor Iain D. Gilchrist.
%    \item \textit{Awarded:}
%      3 March, 2003
    \end{itemize}
  \item M.A. Neuropsychology (distinction), Radboud University Nijmegen, 1999.
\end{itemize}

\section*{Employment}
\begin{itemize}
% \item Professor of Cognitive Science, School of Psychological Science, University of Bristol, August 2023.
\item Reader in Experimental Psychology (Associate Professor), School of Psychological Science, University of Bristol, August 2012 (proleptic appointment).
\item EPSRC Advanced Research Fellow (promoted to Senior Research Fellow in 2011), School of Experimental Psychology, University of Bristol, March 2008--February 2013.
\item Research Associate (promoted to Research Fellow in 2006) School of Experimental Psychology, University of Bristol, November 2004--February 2008.
\item Post-doctoral Research Assistant, School of Experimental Psychology, University of Bristol,
  November 2002--October 2004.
\end{itemize}

%\section*{Research statement}
%\begin{itemize} %just using itemisation here to bring the indentation in line with the rest of the CV
%\item I am a cognitive scientist with a core interest in decision making at various levels of complexity (from basic sensorimotor actions to high-level, cognitive decisions). Real-world environments are characterised by uncertainty. We deal with this uncertainty by sampling and integrating information (across sources, space, time) and by combining noisy information with prior beliefs. These basic mechanisms are at play in many real-world domains such as consumer choice, drawing inferences from polling data, or performing some complex visual-motor routine. My current research interests are: action sequencing in complex, naturalistic behaviour; models of fixation control; inferring mental states from movement; learning and decision making; data visualisation ``literacy''. %I am planning research proposals to investigate how parameters of cognitive models evolve over the course of behaviour;
%	% We develop computational models that can be used to derive specific, quantitative predictions about cognition and behaviour. 
%\end{itemize}

\section*{Peer-reviewed publications}
%\begin{itemize}
%\item \center \textbf{Four currently nominated articles for the upcoming Research Excellence Framework are marked with a *.}
%\end{itemize}

\begin{itemize}
\item Han, Q., Quadflieg, S., \& Ludwig, C.J.H. (2023). Decision avoidance and post-decision regret: a systematic review and meta-analysis. \textit{PLoS ONE}, in press.
\item Hadjipanayi, V., Ludwig, C.J.H., \& Kent, C. (2023). Graded prioritisation of targets in search: Reward diminishes the low prevalence effect. \textit{Cognitive Research: Principles and Implications}, \textit{8}, 52, \href{https://doi.org/10.1186/s41235-023-00507-9}{doi:10.1186/s41235-023-00507-9}.
\item Ez-zizi, A., Farrell, S., Leslie, D., Malhotra, G., \& Ludwig, C.J.H. (2023). Reinforcement learning under uncertainty: Expected versus unexpected uncertainty and state versus reward uncertainty. \textit{Computational Brain \& Behavior}, \href{https://doi.org/10.1007/s42113-022-00165-y}{doi:10.1007/s42113-022-00165-y}.
\item Montera, M.L., Bowers, J., Ponte-Costa, R., Ludwig, C.J.H., \& Malhotra, G. (2022). Lost in latent space: Examining failures of disentangled models at combinatorial generalisation. \textit{Neural Information Processing Systems}.
\item Hadjipanayi, V., Shimi, A., Ludwig, C.J.H., \& Kent, C. (2022). Unequal allocation of overt and covert attention in Multiple Object Tracking. \textit{Attention, Perception, \& Psychophysics}, \textit{84}, 1519--1537, \href{https://doi.org/10.3758/s13414-022-02501-7}{doi:10.3758/s13414-022-02501-7}.
\item Santangelo, A.P., Ludwig, C.J.H., Navajas, J., Sigman, M., \& Leone, M.J. (2022). Background music changes the policy of human decision-making: Evidence from experimental and drift-diffusion model-based approaches on different decision tasks. \textit{Journal of Experimental Psychology: General}, \textit{151}, 2222--2236, \href{https://doi.apa.org/doi/10.1037/xge0001189}{doi: 10.1037/xge0001189}.
\item Stewart, E.E.M., Ludwig, C.J.H., \& Sch\"{u}tz, A.C. (2022). Humans represent the precision and utility of information acquired across fixations. \textit{Scientific Reports}, \textit{12}(1): 1--13, \href{https://doi.org/10.1038/s41598-022-06357-7}{doi:10.1038/s41598-022-06357-7}.
\item Sullivan, B., Ludwig, C.J.H., Damen, D., Mayol-Cuevas, W., \& Gilchrist, I.D. (2021). Look-ahead fixations during visuomotor behavior: Evidence from assembling a camping tent. \textit{Journal of Vision}, \textit{21}: 13, \href{https://doi.org/10.1167/jov.21.3.13}{doi: 10.1167/jov.21.3.13}.
\item Blything, R., Biscione, V., Vankov, I., Ludwig, C.J.H., \& Bowers, J.S. (2021). The human visual system and CNNs can both support robust online translation tolerance following extreme displacements. \textit{Journal of Vision}, \textit{21}: 9, \href{https://doi.org/10.1167/jov.21.2.9}{doi: 10.1167/jov.21.2.9}.
\item Montera, M.L., Ludwig, C.J.H., Ponte-Costa, R., Malhotra, G., Bowers, J. (2021). The role of disentanglement in generalisation. \textit{International Conference on Learning Representations},\\\href{https://openreview.net/forum?id=qbH974jKUVy}{https://openreview.net/forum?id=qbH974jKUVy}
\item Attwood, A.S., Ludwig, C.J.H., Penton-Voak, I.S., Poh, J., Kwong, A.S.F., \& Munaf{\`o}, M.R. (2021). Effects of state anxiety on gait: A 7.5\% carbon dioxide challenge study. \textit{Psychological Research}, doi: 10.1007/s00426-020-01393-2.
\item Fradkin, I., Ludwig, C., Elder, E., \& Huppert, J.D. (2020). Doubting what you already know: uncertainty regarding state transitions is associated with obsessive compulsive symptoms. \textit{PLoS Computational Biology}, \textit{16}(2): e1007634, doi: 10.1371/journal.pcbi.1007634.
\item Xiong, C., Ceja, C., Ludwig, C.J.H., \& Franconeri, S. (2020). Biased Average Position Estimates in Line and Bar Graphs: Underestimation, Overestimation, and Perceptual Pull. \textit{IEEE Transactions on Visualization and Computer Graphics}, \textit{26}, 301--310, doi: 10.1109/TVCG.2019.2934400.
\item Jang, Y., Sullivan, B., Ludwig, C., Gilchrist, I., Damen, D., \& Mayol-Cuevas, W. (2019). EPIC-tent: An egocentric video dataset for camping tent assembly. \textit{ICCV Workshop on Egocentric Perception, Interaction and Computing}.
\item Malhotra, G., Leslie, D.S., Ludwig, C.J.H., \& Bogacz, R. (2018). Time-varying decision boundaries: insights from optimality analysis. \textit{Psychonomic Bulletin \& Review}, \textit{25}, 971--996, doi: 10.3758/s13423-017-1340-6.
\item Ludwig, C.J.H., Alexander, N., Howard, K.L., Jedrzejewska, A.A., Mundkur, I., Redmill, D.W. (2018). The influence of visual flow and perceptual load on locomotion speed. \textit{Attention, Perception, \& Psychophysics}, \textit{80}, 69--81, doi: 10.3758/s13414-017-1417-3.
\item M\'{e}gardon, G., Ludwig, C.J.H., \& Sumner, P. (2017). Trajectory curvature in saccade sequences: spatiotopic influences vs residual motor activity. \textit{Journal of Neurophysiology}, \textit{117}, 1310--1320, doi: 10.1152/jn.00110.2017.
\item Malhotra, G., Leslie, D.S., Ludwig, C.J.H., \& Bogacz, R. (2017). Overcoming indecision by changing the decision boundary. \textit{Journal of Experimental Psychology: General}, \textit{146}, 776--805, \\ doi: 10.1037/xge0000286.
\item Mason, A., Farrell, S, Howard-Jones, P., \& Ludwig, C.J.H. (2017). The role of reward and reward uncertainty in episodic memory. \textit{Journal of Memory and Language}, \textit{96}, 62--77, \\doi: 10.1016/j.jml.2017.05.003.
\item Mason, A., Ludwig, C., \& Farrell, S. (2017). Adaptive scaling of reward in episodic memory: A replication study. \textit{Quarterly Journal of Experimental Psychology}, \textit{70}, 2306--2318, \\doi: 10.1080/17470218.2016.1233439.
\item Ludwig, C.J.H., \& Evens, D.R. (2017). Information Foraging for Perceptual Decisions. \textit{Journal of Experimental Psychology: Human Perception \& Performance}, \textit{43}, 245-264, doi: 10.1037/xhp0000299.
\item Bowers, J.S., Vankov, I.I., \& Ludwig, C.J.H. (2016). The visual system supports on-line translation invariance for object identification. \textit{Psychonomic Bulletin \& Review}, \textit{23}, 432--438, doi: 10.3758/s13423-015-0916-2.
\item Ludwig, C.J.H., Davies, J.R., \& Eckstein, M.P. (2014). Foveal analysis and peripheral selection during active visual sampling. \textit{Proceedings of the National Academy of Sciences}, \textit{111}(2), E291-E299, doi: 10.1073/pnas.1313553111.
\item Cassey, T.C., Evens, D.R., Bogacz, R., Marshall, J.A.R., \& Ludwig, C.J.H. (2013). Adaptive Sampling of Information in Perceptual Decision-Making. \textit{PLoS ONE}, \textit{8}(11): e78993, doi: 10.1371/journal.pone.0078993.
\item Ludwig, C.J.H., Davies, J.R., \& Gegenfurtner, K.R. (2012). A functional role for trans-saccadic luminance differences. \textit{Journal of Vision}, \textit{12}(13):14, 1-21, doi: 10.1167/12.13.14.
\item Ludwig, C.J.H., Farrell, S., Ellis, L.A., Hardwicke, T.E., \& Gilchrist, I.D. (2012). Context-gated statistical learning and its role in visual-saccadic decisions. {\it Journal of Experimental Psychology: General, 141}, 150-169, doi: 10.1037/a0024916.
\item Ludwig, C.J.H., \& Davies, J.R. (2011). Estimating the growth of internal evidence guiding perceptual decisions. {\it Cognitive Psychology, 63}, 61-92, doi: 10.1016/j.cogpsych.2011.05.002.
\item Etchells, P.J., Ludwig, C.J.H., Benton, C.P., \& Gilchrist, I.D. (2011). Testing a simplified method for measuring velocity integration in saccades using a manipulation of target contrast. {\it Frontiers in Psychology, 2}, 115, doi: 10.3389/fpsyg.2011.00115.
\item Ludwig, C.J.H. (2011). Saccadic decision-making. In S.P. Liversedge, I.D. Gilchrist, and S. Everling (Eds.), {\it The Oxford Handbook of Eye Movements.} Oxford: OUP.
\item Farrell, S., Ludwig, C.J.H., Ellis, L.A., \& Gilchrist, I.D. (2010). The influence of environmental statistics on inhibition of saccadic return. {\it Proceedings of the National Academy of Sciences, 107}, 929-934, doi: 10.1073/pnas.0906845107.
\item Evens, D.R., \& Ludwig, C.J.H. (2010). Dual-task costs and benefits in anti-saccade performance. {\it Experimental Brain Research, 4}, 545-557, doi: 10.1007/s00221-010-2393-1.
\item Etchells, P.J., Benton, C.P., Ludwig, C.J.H., \& Gilchrist, I.D. (2010). The target velocity integration function for saccades. {\it Journal of Vision, 10}, 7, doi: 10.1167/10.6.7.
\item Ludwig, C.J.H. (2009). Temporal integration of sensory evidence for saccade target selection. {\it Vision Research, 49}, 2764-2773, doi: 10.1016/j.visres.2009.08.012.
\item Ludwig, C.J.H., Farrell, S., Ellis, L.A., \& Gilchrist, I.D. (2009). The mechanism underlying inhibition of saccadic return. {\it Cognitive Psychology, 59}, 180-202, doi: 10.1016/j.cogpsych.2009.04.002.
\item Cooper, R.M., Rowe, A.C., Penton-Voak, I.S., \& Ludwig, C.J.H. (2009). No reliable effects of emotional facial expression, adult attachment orientation, or anxiety on the allocation of visual attention in the spatial cueing paradigm. {\it Journal of Research in Personality, 43}, 643-652, doi: 10.1016/j.jrp.2009.03.005.
\item Stephen H. Butler, Stephanie Rossit, Iain D. Gilchrist, Casimir J.H. Ludwig, Bettina Olk, Keith Muir, Ian Reeves, \& Monika Harvey (2009). Non-lateralised deficits in anti-saccade performance in patients with hemispatial neglect. {\it Neuropsychologia, 47}, 2488-2495. \\doi: 10.1016/j.neuropsychologia.2009.04.022.
\item Ludwig, C.J.H., Butler, S.H., Rossit, S., Harvey, M., \& Gilchrist, I.D. (2009). Modelling contralesional movement slowing after unilateral brain damage. {\it Neuroscience Letters, 452}, 1-4, doi: 10.1016/j.neulet.2009.01.033.
\item Skelton, R., Ludwig, C.J.H., \& Mohr, C. (2009). A novel, illustrated questionnaire to distinguish projector and associator synaesthetes. {\it Cortex, 45}, 721-729, doi: 10.1016/j.cortex.2008.02.006.
\item Farrell, S., \& Ludwig, C.J.H. (2008). Bayesian and maximum likelihood estimation of hierarchical response time models. {\it Psychonomic Bulletin \& Review, 15}, 1209-1217, doi: 10.3758/PBR.15.6.1209.
\item Ludwig, C. J. H., Ranson, A., \& Gilchrist, I. D. (2008). Oculomotor capture by transient events: A comparison of abrupt onsets, offsets, motion, and flicker. {\it Journal of Vision, 8}, 11. doi: 10.1167/8.14.11.
\item  Ludwig, C.J.H., Mildinhall, J.W., \& Gilchrist, I.D. (2007). A population coding account for systematic variation in saccadic dead time. {\it Journal of Neurophysiology, 97}, 795-805, doi: 10.1152/jn.00652.2006.
\item Ludwig, C.J.H., Eckstein, M.P., \& Beutter, B.R. (2007). Limited flexibility in the filter underlying saccadic targeting. {\it Vision Research, 47}, 280-288, doi: 10.1016/j.visres.2006.09.009.
\item Butler, S.H., Gilchrist, I.D., Ludwig, C.J.H., Muir, K., \& Harvey, M. (2006). Impairments of oculomotor control in a patient with a right temporo-parietal lesion. {\it Cognitive Neuropsychology, 23}, 990-999, doi: 10.1080/13594320600768847.
\item Ludwig, C.J.H., \& Gilchrist, I.D. (2006). The relative contributions of luminance contrast and task demands on saccade target selection. {\it Vision Research, 46}, 2743-2748, doi: 10.1016/j.visres.2006.02.012.
\item Ludwig, C.J.H., Gilchrist, I.D., McSorley, E., \& Baddeley, R.J. (2005). The temporal impulse response underlying saccadic decisions. {\it The Journal of Neuroscience, 25}, 9907-9912, \\doi: 10.1523/JNEUROSCI.2197-05.2005.
\item Peers, P.V., Ludwig, C.J.H., Rorden, C., Cusack, R., Bonfiglioli, C., Bundesen, C., Driver, J., Antoun, N., \& Duncan, J. (2005). Attentional functions of parietal and frontal cortex. {\it Cerebral Cortex, 15}, 1469-1484, doi: 10.1093/cercor/bhi029.
\item Ludwig, C.J.H., Gilchrist, I.D., \& McSorley, E. (2005). The remote distractor effect in saccade programming: Channel interactions and lateral inhibition. {\it Vision Research, 45}, 1177-1190, doi: 10.1016/j.visres.2004.10.019.
\item Ludwig, C.J.H., Gilchrist, I.D., \& McSorley, E. (2004). The influence of spatial frequency and contrast on saccade latencies. {\it Vision Research, 44}, 2597-2604, doi: 10.1016/j.visres.2004.05.022.
\item Ludwig, C.J.H., \& Gilchrist, I.D. (2003). Goal-driven modulation of oculomotor capture. {\it Perception \& Psychophysics, 65}, 1243-1251.
\item Ludwig, C.J.H. \& Gilchrist, I.D. (2003). Target similarity affects saccade curvature away from irrelevant onsets. {\it Experimental Brain Research, 152}, 60-69, doi: 10.1007/s00221-003-1520-7.
\item Ludwig, C.J.H. \& Gilchrist, I.D. (2002). Measuring saccade curvature: A curve-fitting approach. {\it Behavior Research Methods, Instruments, and Computers, 34}, 618-624.
\item Ludwig, C.J.H. \& Gilchrist, I.D. (2002). Stimulus-Driven and Goal-Driven Control over Visual Selection. {\it Journal of Experimental Psychology: Human Perception and Performance, 28}, 902-912, doi: 10.1037/0096-1523.28.4.902.
\end{itemize}

%\section*{Work under review/in progress}
%\begin{itemize}
%	\item Malhotra, G., Gilchrist, I.D., \& Ludwig, C.J.H. (in progress). Distinguishing between evidence integration and independent sampling in models of decision making. (target journal: Psychological Review)
%	\item Chapman, W., Gilchrist, I.D., \& Ludwig, C.J.H. (in progress). Decision uncertainty in reaching and grasping. (target journal: Journal of Experimental Psychology: Human Perception \& Performance)
%	\item Han, Q, Quadflieg, S., \& Ludwig, C.J.H. (in progress). The structure of information provision affects current and future decision strategy utilisation. (target journal: Journal of Behavioral Decision Making)
%	\item Ludwig, C.J.H. (in progress). Foveal and peripheral control over fixation duration. (target journal: Journal of Experimental Psychology: General)
%	\item Ludwig, C.J.H., Trukenbrod, H.A., \& Engbert, R. (in progress). Sequential adjustment of fixation duration to changes in foveal processing load. (target journal: Journal of Vision)
%	\item Ludwig, C.J.H., \& Trukenbrod, H.A. (in progress). FEAT: Fixation control by Accumulation of Evidence to Threshold. (target journal: Computational Brain \& Behavior).
%\end{itemize}

\section*{Grants}
\begin{itemize}
\item Ludwig, C.J.H. (2021--2023). Likelihood-free dynamical models of sequential fixation control. \textit{Leverhulme Trust Research Fellowship}, \pounds50k.
\item Pearson, R., Lawlor, D., Ludwig, C., Skinner, A., Tilling, K., Caldwell, D., Culpin, I., \& Jones, H. (2017-2022). Genetic, behavioural and cognitive mechanisms underpinning the association between mother and offspring mental health problems: mental (M) health (H) intergenerational transmission (INT) -(MHINT). \textit{Horizon 2020 ERC}, \pounds1.18m.
\item Mayol-Cuevas, W., Damen, D., Gilchrist, I., \& Ludwig, C. (2016-2020). GLANCE: GLAnceable Nuances for Contextual Events. \textit{Engineering and Physical Sciences Research Council}, \pounds807k.
\item Bull, D., Burn, J., Canagarajah, N., Cuthill, I., Gilchrist, I., Ludwig, C., \& Roberts, N. (2015--2020). Vision for the future. \textit{Engineering and Physical Sciences Research Council}, \pounds1.4m.
\item Gilchrist, I.D., Leslie, D., Baddeley, R., Bogacz, R., Farrell, S., Ludwig, C.J.H., \& McNamara, J. (2011--2015). Decision-making in an unstable world. {\it Engineering and Physical Sciences Research Council}, \pounds1.6m.
\item Ludwig, C.J.H., Burn, J.F., Leonard, U., \& Bull, D.R. (2010--2014). Bristol Vision Institute Laboratory. {\it The Wellcome Trust}, equipment grant, \pounds390k.
\item Ludwig, C.J.H. (2008--2013). Integrating 'when' and 'where' in models of saccade target selection. {\it Engineering \& Physical Sciences Research Council}, Advanced Research Fellowship, \pounds700k.
\item Ludwig, C.J.H. \& Gilchrist, I.D. (2006--2007). The salience of luminance transients to the saccadic eye movement system. {\it Engineering \& Physical Sciences Research Council}, \pounds120k.
\item Farrel, S., Ludwig, C.J.H., Gilchrist, I.D., \& Carpenter, R.H.S. (2006--2009). Modelling sequential effects in saccadic choice. {\it The Wellcome Trust}, \pounds119k.
\item Rowe, A.C., Penton-Voak, I.S., \& Ludwig, C.J.H. (2006--2009). Adult attachment and the perceptual processing of facial expressions of emotion. {\it Economic \& Social Research Council}, \pounds370k.
\item Ludwig, C.J.H. (2005). The perceptual template governing saccadic decisions. {\it Engineering \& Physical Sciences Research Council}, Overseas Travel Grant, \pounds6k.
\end{itemize}

\section*{Visiting positions}
\begin{itemize}
	\item Visiting Scientist, University of Potsdam, SFB 1294 grant on Data Assimilation (Ralf Engbert \& Sebastian Reich), 2018.
	\item Visiting Scientist, Universidad Torcuato di Tella, Buenos Aires, Argentina, 2015
	\item Engineering \& Physical Sciences Research Council (EPSRC), Advanced Research Fellowship, 2008--2013.
	\item Visiting Scientist, University of California Santa Barbara, US, 2005, 2011.
	\item Medical Research Council---Cognition and Brain Sciences Unit, Summer Studentship, 1999.
\end{itemize}

\section*{Teaching and supervision}
\begin{itemize}
\item Lecturing
    \begin{itemize}
    \item Undergraduate
    	\begin{itemize}
    	\item Year 2 Brain \& Cognition (2019--present).
    	\item Year 3 Contemporary Issues/Current Topics in Psychology (2013--present).
    	\item Year 3 Dissertation projects (2012--present).
    	\item Year 2 Biological Psychology (2012--2019).
    	\item Year 1 Biological Psychology (2012--2014).
    	\end{itemize}
    \item Post-graduate: MSc 
    	\begin{itemize}
    	\item Brain \& Cognition (MSc Conversion; 2019--present).
    	\item Dissertation projects (2012--present).
    	\item Biological Psychology (MSc Conversion; 2017--2019).
    	\item Academic careers sessions (2016--2018).
		\item Neuropsychological Analysis Tools (2014--2017): Recording and analysing movements.    	
    	\item Generic Research Skills (2003--2016): Ethics; APA style; Academic careers.
   	 	\item Functional Neuroanatomy and Neuroscience Methods (2009--2010): Eye movements.
    	\item Foundations of Vision (2008--2009): Eye movements.
    	\end{itemize}
    \end{itemize}
\item{PhD supervision}
    \begin{itemize}
    \item Michele Garibbo, 2020--present (co-supervisor, Wellcome Trust-funded). 
    \item Will Chapman, 2016--2023 (EPSRC-funded).
    \item Veronika Hadjipanayi, 2019--2023 (co-supervisor).
    \item Ilaria Costantini, 2018--2023 (co-supervisor, ERC-funded)
    \item Qing Han, 2017--2022 (part UoB-funded).
    \item Alice Mason, 2011--2016 (EPSRC-funded).
    \item David Evens, 2009--2014 (EPSRC-funded).
    \item Peter Etchells, 2007--2010 (co-supervisor).
    \end{itemize}
\item{MSc by Research supervision}
	\begin{itemize}
		\item Erik Stuchl\'y, 2020--2021.
		\item Joe Wilson, 2019--2021 (co-supervisor).
	\end{itemize}
\item Miscellaneous
	\begin{itemize}
	\item PhD examination: Onyeka Amiebenomo (University of Cardiff, 2023); Lisa Schwetlick (University of Potsdam, 2023); Anne-Lene Sax (University of Bristol, 2022); Chloe Slaney (University of Bristol, 2020); Vikki Neville (University of Bristol, 2019); Solveiga Stonkute (University of Cardiff, 2019); C{\'e}cille Vullings (Universit\'e de Lille, 2018); Hildward Vandormael (University of Oxford, 2017); Adnane Ez-Zizi (University of Bristol, 2016); Stephen Hinde (University of Bristol, 2017); Bobby Stuijfzand (University of Bristol, 2017); Rob Udale (University of Bristol, 2017); Yi Shin-Lin (University of Birmingham, 2015).
	\item MSc by Research examination: Stella Becci (University of Bristol, 2019); Katie Joyce (University of Bristol, 2017).
    \item Lab demonstrator throughout my undergraduate degree and PhD on units related to IT skills and statistics (1997--2002).
	\end{itemize}
\end{itemize}

\section*{Administration}
\begin{itemize}
\item Admissions Officer - UG Recruitment (2022--present).
\item Unit director year 2 / MSc Brain \& Cognition (2019--present).
\item Year 2 coordinator (2019--2021).
\item Deputy School Education Director (PGT focus; 2016--2020).
\item Programme director MSc Research Methods (2014--2017).
\item Programme director Msc Neuropsychology (2013--2017).
\item Student Placement co-ordinator (2012--present).
\item Unit director Biological Psychology (year 2 BSc 2013--2019; MSc Conversion 2017--2019).
\item Member of the Faculty of Engineering Ethics Committee (2009--2011).
\item Member of the Faculty of Science Human Research Ethics Committee (2004--2014).
\item Unit director year 1 Biological Psychology (2012--2013).
\end{itemize}

\section*{Invited talks at conferences/workshops}
\begin{itemize}
\item Keynote at `Model-based Neuroscience' Summer School, Amsterdam, 2023; Workshop on `From Peripheral to Transsaccadic and Foveal Perception', Rauischholzhausen, Germany, 2019; Workshop on `Cognitive and Motor Processes in Visual Attention', Durham, UK, 2019; MRC Doctoral Training Programme workshop on `Computational Modelling', UK, 2018; Royal Society meeting on `Understanding images in biological and computer vision', UK, 2018; Workshop on `Learning at the Interface of Vision and Oculomotor Control', Humboldt University Berlin, Germany, 2016; DSTL Knowledge Network Meeting, Winfrith Technology Centre, Winfrith, UK, 2008; AGM of the European Brain and Behaviour Society, Dublin, Ireland, 2005.
\end{itemize}

\section*{Invited seminars}
\begin{itemize}
\item University of Potsdam, Germany, 2022, 2017, 2012; Humboldt University Berlin, Germany, 2022, 2018; Paris Descartes University, France, 2017; University of Cardiff, UK, 2016; University of Bielefeld, Germany, 2016; Aston University, UK, 2015; Universidad Torcuato Di Tella, Buenos Aires, Argentina, 2015; University of Wales, Bangor, 2014; University of California Irvine, 2014; University of Southampton, UK, 2014; University of Giessen, Germany, 2009; University of Birmingham, UK, 2009; University of Strathclyde, UK, 2009; University of Geneva, Switzerland, 2008; University of Cardiff, UK, 2008; University of Leicester, UK, 2007; Smith-Kettlewell Eye Research Institute, US, 2005; NASA Ames Research Centre, US, 2005; University of Nottingham, 2003; Royal Holloway, University of London, UK, 2002; Medical Research Council -- Cognition \& Brain Sciences Unit, 2000.
\end{itemize}

\section*{Professional activity}
\begin{itemize}
\item External Examiner MSc Computational Neuroscience \& Cognitive Robotics; MSc Brain Imaging and Cognitive Neuroscience; MSc Psychology at University of Birmingham (2019--2023).
\item Associate Editor of The Quarterly Journal of Experimental Psychology (2017--2020).
\item Member of the EPSRC college (2008--present).
\item Reviewer for {\it ACM Transactions on Applied Perception; Attention, Perception, \& Psychophysics; Behavior Research Methods; Brazilian Journal of Medical and Biological Research; Cognition; Computational Brain \& Behavior; Cortex; Ergonomics; Experimental Brain Research; Frontiers in Neuroscience; Journal of Cognitive Neuroscience; Journal of Experimental Psychology: General; Journal of Experimental Psychology: Human Perception \& Performance; Journal of Mathematical Psychology; Journal of Neurophysiology; Journal of Social and Personal Relationships; Journal of Vision; Perception/iPerception; PLoS Computational Biology; PLoS ONE; Proceedings of the National Academy of Sciences; Proceedings of the Royal Society B: Biological Sciences; Psychological Review; Psychological Science; Psychonomic Bulletin \& Review; Psychopharmacology; Quarterly Journal of Experimental Psychology; Spatial Vision; Vision Research; Visual Cognition.}
\item Reviewer for \textit{Agence National Recherche; Biotechnology \& Biological Sciences Research Council; British Academy; Economic \& Social Research Council; Engineering \& Physical Sciences Research Council; European Research Council (ERC); Flanders Research Foundation (FWO); Help the Aged; The Leverhulme Trust; Medical Research Council; National Science Foundation; Swiss National Science Foundation; Wellcome Trust}.
\item Instructor on the Summer School on Computational Modelling of Cognition, 2014--present (bi-annual).
\item Member of Applied Vision Association, British Machine Vision Association, Society for Mathematical Psychology, Experimental Psychology Society.
\item Fellow of the Psychonomic Society.
\item Languages: Dutch (native); English (native level); German (basic).
\end{itemize}

\newpage
\appendix

\section*{Appendix}

\subsection*{Miscellaneous publications}
\begin{itemize}
\item Ludwig, C.J.H. (2002). Visual attention and cortical circuits. {\it Perception, 31}, 513-514. (book review)
\end{itemize}

\subsection*{Peer-reviewed conference presentations}
%Select either the synopsis below or the full list of presentations
\begin{itemize}
\item Ludwig, C., Stuchl\'y, E., \& Malhotra, G. (2023). Navigating cognitive parameter space. \textit{Annual meeting Society for Mathematical Psychology}. 
\item Stuchl\'y, E., Ludwig, C., \& Malhotra, G. (2022). Testing the optimisation hypothesis by tracking changes in decision strategy in a reward maximisation task. \textit{Annual meeting Society for NeuroEconomics}. 
\item Garibbo, M., Ludwig, C., Lepora, N., \& Aitchison, L. (2022). What deep reinforcement learning tells us about human motor learning and vice-versa. \textit{Cold Spring Harbor Laboratory meeting: From Neuroscience to Artificially Intelligent}.
\item Stuchl\'y, E., Malhotra, G., \& Ludwig, C. (2022). Defining one’s boundaries: Exploration of the decision threshold parameter space during a reward maximisation task. \textit{Annual meeting Society for Mathematical Psychology} (virtual).
\item Chapman, W., \& Ludwig, C. (2022). The effects of perceptual uncertainty on reach to grasp movements. \textit{Annual Meeting Vision Sciences Society}.
\item Hadjipanayi, V., Ludwig, C.J.H., \& Kent, C. (2021). Unequal allocation of spatial attention in Multiple Object Tracking. \textit{European Conference on Visual Perception} (virtual).
\item Stuchl\'y, E., Ludwig, C., Malhotra, G. (2021). Not great, not terrible: A reward ``landscape'' analysis of time-varying decision thresholds. \textit{Annual meeting Society for Mathematical Psychology} (virtual).
\item Garibbo, M., Ludwig, C., Lepora, N., \& Aitchison, L. (2021). What can deep reinforcement learning tell us about human motor learning and vice-versa? \textit{Neuromatch Conference 4} (virtual).
\item Stewart, E.E.M., Ludwig, C.J.H., Sch{\"u}tz, A.C. (2021). Humans retain a representation of information acquired across fixations. \textit{TeaP} (Tagung Experimentell Arbeitender Psychologen; virtual).
\item Montero, M.L., Ponte-Costa, R., Bowers, J.S., Ludwig, C., Malhotra, G. (2020). Perceptual front-ends for decision-making. \textit{Annual meeting Society for Mathematical Psychology} (virtual).
\item Stewart, E.E.M., Ludwig, C., Sch{\"u}tz, A.C. (2020). Humans are more confident for items they have fixated before. \textit{Annual meeting Vision Sciences Society} (virtual).
\item Hadjipanayi, V., Kent, C., Ludwig, C. (2020). Eye movements during unequal attention splitting in a multiple object tracking task. \textit{Meeting of the Experimental Psychology Society} (virtual). 
\item Han, Q., Ludwig, C.J.H., Quadflieg, S. (2020). Decision Avoidance and Regret: A systematic review and meta-analysis. \textit{Meeting of the Experimental Psychology Society.}
\item Blything, R. Vankov, I., Ludwig, C.J.H., \& Bowers, J.S. (2019). \textit{Conference on Cognitive Computational Neuroscience}.
\item Sullivan, B., Doughty, H., Mayol-Cuevas, W., Damen, D., Ludwig, C., \& Gilchrist, I. (2019). Predicting self-rated uncertainty from eye movements in a natural task. \textit{European Conference on Visual Perception.}
\item Blything, R. Vankov, I., Ludwig, C.J.H., \& Bowers, J.S. (2019). Translation tolerance in vision. \textit{Proceedings of the Cognitive Science Society}.
\item Ludwig, C.J.H., Trukenbrod, H.A., Engbert, R. (2019). Temporal control in modelling eye fixations. \textit{European Conference on Eye Movements.}
\item Ludwig, C.J.H., Trukenbrod, H.A., Engbert, R. (2019). Temporal control in modelling eye fixations. \textit{Annual meeting Society for Mathematical Psychology.}
\item  Xiong, C., Ceja, C.R., Ludwig, C., \& Franconeri, S. (2019). Attraction and response probe similarity effects in a multiple ensemble judgment task. \textit{Annual meeting Vision Sciences Society.}
\item Sullivan, B., Doughty, H., Mayol-Cuevas, W., Damen, D., Ludwig, C., \& Gilchrist, I. (2018). Detecting uncertainty while assembling a camping tent. \textit{European Conference on Visual Perception.}
\item Malhotra, G., Ludwig, C., \& Gilchrist, I. (2017). Distinguishing between evidence accumulation and temporal probability summation in perceptual decision making. \textit{European Conference on Visual Perception.}
\item Ludwig, C., Alexander, N., Mundkur, I., \& Redmill, D. (2017). The interaction of visual flow and perceptual load in the control of locomotion speed. \textit{European Conference on Visual Perception.}
\item Sullivan, B., Ludwig, C., Gilchrist, I., Damen, D., Mayol-Cuevas, W. (2017). Predicting eye and head coordination while looking and pointing. \textit{European Conference on Visual Perception.}
\item Malhotra, G., Ludwig, C., \& Gilchrist, I. (2017). Distinguishing between evidence accumulation and temporal probability summation in perceptual decision making. \textit{Annual meeting Society for Mathematical Psychology.}
\item Ludwig, C.J.H., Stothart, G., Hedge, C., \& Skinner, A. (2015). Control of fixation duration by foveal and peripheral evidence. \textit{European Conference on Eye Movements.}
\item Jarvstad, A., Ludwig, C.J.H., Bogacz, R., \& Gilchrist, I.D. (2015). Expectation and saccade latency: Prior probability and number of alternatives. \textit{European Conference on Eye Movements.}
\item Megardon, G., Sumner, P., \& Ludwig, C.J.H. (2015). Spatio-temporal stimuli interactions in visual-oculomotor pathway: Are they spatiotopic or retinotopic? \textit{European Conference on Eye Movements.}
\item Ludwig, C., \& Evens, D. (2014). Information foraging for perceptual decisions. \textit{Psychonomics Society's 55th Annual Meeting.}
\item Malhotra, G., Leslie, D., Ludwig, C., \& Bogacz, R. (2014). Changing decision criteria in rapid and slow decisions: Do people behave optimally? \textit{Decision-Making Bristol.}
\item Mason, A., Ludwig, C., Howard-Jones, P., \& Farrell, S. (2014). Chance-based uncertainty of reward improves long-term memory. \textit{Decision-Making Bristol}.
\item Ludwig, C., \& Evens, D. (2014). Information foraging for perceptual decisions. \textit{Decision-Making Bristol.}
\item Ludwig, C.J.H., Davies, J.R., \& Eckstein, M.P. (2013). Parallel and independent foveal analysis and peripheral selection. {\it European Conference on Eye Movements.}
\item Ludwig, C.J.H., Davies, J.R., \& Eckstein, M.P. (2012). Parallel extraction of information for foveal analysis and peripheral selection of where to look next. {\it Annual meeting Vision Sciences Society.}
\item Evens, D.R., Cassey, T., Marshall, J.A.R., Bogagz, R., \& Ludwig, C.J.H. (2012). Active visual sampling in uncertain environments. {\it Annual meeting Vision Sciences Society.}
\item Ludwig, C.J.H., Farrell, S., Ellis, L.A., Hardwicke, T.E., \& Gilchrist, I.D. (2011). Context-selective belief-updating accounts for "noise" in accumulator models of saccadic choice and latency. {\it European Conference on Eye Movements.}
\item Davies, J.R., \& Ludwig, C.J.H. (2011). The eyes are driven by visual mechanisms that receive novel inputs across saccades. {\it European Conference on Eye Movements.}
\item Evens, D.R., Cassey, T., Marshall, J.A.R., Bogagz, R., \& Ludwig, C.J.H. (2011). Active visual sampling strategy adapts to environmental uncertainty. {\it European Conference on Eye Movements.}
\item Cassey, T., Evens, D.R., Ludwig, C.J.H., Marshall, J.A.R., \& Bogacz, R. (2011). Adaptive sampling of information during perceptual decision problems. {\it Annual meeting Society for Mathematical Psychology.}
\item Davies, J.R., \& Ludwig, C.J.H. (2011). Saccadic decisions in response to new objects in spatiotopic and retinotopic reference frames. {\it Annual meeting Vision Sciences Society.}
\item Ludwig, C.J.H., \& Davies, J.R. (2011). Estimating the growth of discriminative information guiding perceptual decisions. {\it Annual meeting Vision Sciences Society.}
\item Etchells, P.J., Benton, C.P., Ludwig, C.J.H., \& Gilchrist, I.D. (2010). Contrast effects on velocity integration for saccades to moving targets. {\it European Conference on Visual Perception.}
\item Ludwig, C.J.H., \& Farrell, S. (2010). Adaptation of oculomotor IOR to the environmental statistics through sequential updating of expectations. {\it European Conference on Visual Perception.}
\item Ludwig, C.J.H., Farrell, S., Ellis, L.A., Hardwicke, T.E., \& Gilchrist, I.D. (2010). Statistical learning facilitates adaptive eye movement decisions. {\it European Conference on Visual Perception.}
\item Ludwig, C.J.H., Farrell, S., Ellis, L.A., \& Gilchrist, I.D. (2009). Inhibition of saccadic return is sensitive to the probabilistic context. {\it European Conference on Eye Movements.}
\item Evens, D.R., Gilchrist, I.D., \& Ludwig, C.J.H. (2009). Anti-saccades do not benefit from attentional diversion. {\it European Conference on Eye Movements.}
\item Ellis, L.A., Farrell, S., Ludwig, C.J.H., \& Gilchrist, I.D. (2009). The modulation of inhibition of saccadic return by environmental statistics: Is there a role for predictive cues? {\it European Conference on Eye Movements.}
\item Etchells, P.J., Benton, C.P., Ludwig, C.J.H., \& Gilchrist, I.D. (2009). Estimation of temporal motion integration in saccades to moving targets. {\it European Conference on Eye Movements.}
\item Ludwig, C.J.H., Farrell, S., Ellis, L.A., \& Gilchrist, I.D. (2009). Inhibition of saccadic return is sensitive to the probabilistic
      structure of the environment. {\it Annual meeting Vision Sciences Society.}
\item Etchells, P.J., Benton, C.P., Ludwig, C.J.H., \& Gilchrist, I.D. (2009). Intercepting moving targets: Estimating motion integration
      and saccadic dead time. {\it Annual meeting Vision Sciences Society.}
\item Ludwig, C.J.H. (2008). Modelling saccadic decision-making. {\it Applied Vision Association.}
\item Etchells, P.J., Benton, C.P., Ludwig, C.J.H., \& Gilchrist, I.D. (2008). Hitting a movement target: interaction between `when' and `where' in saccade programming. {\it European Conference on Visual Perception.}
\item Rushton, S.K., \& Ludwig, C.J.H. (2008). Capture of attention by scene-relative object movement during self-movement. {\it European Conference on Visual Perception.}
\item Ludwig, C.J.H., Ranson, A., \& Gilchrist, I.D. (2007). Comparison of the effect of dynamic visual events on saccade target selection. {\it European Conference on Eye Movements.}
\item Ludwig, C.J.H., Ranson, A., Droulias, A., \& Gilchrist, I.D. (2007). Limited temporal integration for saccade generation. {\it European Conference on Eye Movements.}
\item Ludwig, C.J.H., \& Gilchrist, I.D. (2007). A sequential sampling model of saccadic double-steps in direction. {\it Annual meeting Vision Sciences Society.}
\item Ludwig, C.J.H., Eckstein, M.P., \& Beutter, B.R. (2006). Limited flexibility in the filter underlying saccadic targeting. {it European Conference on Visual Perception.}
\item Butler S.H., Ludwig C.J.H., Gilchrist I.D., Muir K., Bone I., Reeves I., Duncan G., \& Harvey M. (2006). Evidence from converging paradigms for posterior cortical involvement in response inhibition. {\it European Conference of Visual Perception.}
\item Harvey M., Butler S.H., Ludwig C.J.H., Muir K., Bone I., Reeves I., Duncan G., \& Gilchrist I.D. (2006). Magno-cellular impairment drives size distortion in hemispatial neglect. {\it European Conference on Visual Perception.}
\item Ludwig, C.J.H., \& Gilchrist, I.D. (2005). Stimulus-driven and goal-driven contributions to capture. {\it AGM of the European Brain and Behaviour Society.}
\item Butler S.H., Ludwig C.J.H., Gilchrist I.D., \& Harvey M. (2005). Parietal lobe involvement in response inhibition. {\it AGM European Brain and Behaviour Society.}
\item Ludwig, C.J.H. , Gilchrist, I.D., McSorley, E., \& Baddeley, R.J. (2005). The temporal impulse response underlying saccadic decisions. {\it European Conference on Eye Movements.}
\item Ludwig, C.J.H., \& Gilchrist, I.D. (2005). Determinants of saccadic choice: Contrast response and task demands. {\it European Conference on Eye Movements.}
\item Butler S., Ludwig C.J.H., Gilchrist I.D., Muir K., \& Harvey M. (2005). Anti-saccades in patients with right parietal lesions. {\it European Conference on Eye Movements.}
\item Ludwig, C.J.H., Gilchrist, I.D., McSorley, E., \& Baddeley, R.J. (2005). Visual sampling and saccadic decisions. {\it Annual meeting Vision Sciences Society.}
\item Ludwig, C.J.H., \& Gilchrist, I.D. (2005). On the mapping between saccade latency and visual processing time. {\it Applied Vision Association.}
\item Ludwig, C.J.H., Gilchrist, I.D., \& McSorley, E. (2004). The remote distractor effect in saccade programming: Channel interactions and lateral inhibition. {\it European Conference on Visual Perception.}
\item Butler S.H., Ludwig C.J.H., Gilchrist I.D., \& Harvey M. (2004). Oculomotor capture in a patient with a unilateral right temporo-parietal lesion. {\it European Conference of Visual Perception.}
\item Ludwig, C.J.H., Gilchrist, I.D., \& McSorley, E. (2004). The remote distractor effect in saccade programming: Channel interactions and lateral inhibition. {\it Annual meeting Vision Sciences Society.}
\item Ludwig, C.J.H., Gilchrist, I.D., \& McSorley, E. (2003). Spatial frequency tuning of the saccadic system. {\it European Conference on Eye Movements.}
\item Ludwig, C.J.H., \& Gilchrist, I.D. (2003). Target similarity affects saccade curvature away from irrelevant onsets. {\it European Conference on Eye Movements.}
\item Ludwig, C.J.H., \& Gilchrist, I.D. (2003). Target similarity affects saccade curvature away from irrelevant onsets. {\it Munich Visual Search Symposium.}
\item Ludwig, C.J.H., \& Gilchrist, I.D. (2002). Goal-driven modulation of oculomotor capture. {\it Annual meeting Cognitive Neuroscience Society.}
\item Peers, P., Ludwig, C., Rorden, C., Cusack, R., Driver, J., Bundesen, C., \& Duncan, J. (2002). Using Bundesen's theory of visual attention to quantify attentional deficits following unilateral brain injury. {\it Annual meeting Cognitive Neuroscience Society.}
\item Ludwig, C.J.H., \& Gilchrist, I.D. (2001). Stimulus-Driven and Goal-Driven Control overSaccade Target Selection. {\it European Conference on Eye Movements.}
\item Ludwig, C. J. H., \& Gilchrist, I. D. (2001). Stimulus-driven and goal-driven control over visual selection. {\it Experimental Pychology Society meeting.}
\end{itemize}

\end{document}

%\bigskip
% Footer
%\begin{center}
%  \begin{footnotesize}
%    Last updated: \today \\
%    \href{\footerlink}{\texttt{\footerlink}}
%  \end{footnotesize}
%\end{center}


